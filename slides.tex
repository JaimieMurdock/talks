\documentclass{beamer}
\usetheme{CambridgeUS}

\usepackage{minted}
\usepackage{colortbl}

\title{Practical Parallelism}
\author{Jaimie Murdock}
\institute[IU COGS]{
    IU Cognitive Science Program\\
    810 Eigenmann Hall\\
    \texttt{jammurdo@indiana.edu}
    }
\date{November 17, 2011}

\begin{document}
% Title Page
\frame{\titlepage}
\frame{\tableofcontents}

\section{Introduction}
\subsection{Motivation}
\begin{frame}
\frametitle{Motivation}
Multiprocessing has moved from HPC-only to SMP
% TODO: Find chart of intel desktop shipments
\pause \bigskip

Cores are cheap:
\tiny{
\begin{itemize}
  \item \textbf{2} -- Intel Celeron E3400 -- \$47
  \item \textbf{4} -- AMD Athlon II X4 631 -- \$90
  \item \textbf{6} -- AMD Phenom II X6 1035T -- \$135
  \item \textbf{8} -- AMD FX-8120 -- \$210
\end{itemize}}

\pause \bigskip

The rise of the cloud --- clusters for everyone
% TODO: Insert Google App Engine, Amazon Web Services, Microsoft Azure
\end{frame}

\subsection{Basics}
\begin{frame}
\frametitle{Painfully Parallel Problems}
(or: how to use C211 in an interview)\\
\pause

Given a function that is both commutative and associative (e.g., $+$ or $*$)\\
\pause
\begin{itemize}
  \item \textbf{Commutative:} $x + y = y + x$ \\
  \item \textbf{Associative:} $(x + y) + z = x + (y + z)$
\end{itemize}

% TODO: fix table
%$MapReduce(+, 5 6 4 1 4 1 2 5 6 2 7 6 3 4 6 1)$
%\begin{tabular}[r|cccc]
%Partition: & 5 6 4 1 & 4 1 2 5 & 6 2 7 6 & 3 4 6 1 \\ 
%Map:       & 16 & 12 & 21 & 14 \\
%Reduce:    & 63 & & & \\
%\end{tabular}
\end{frame}

\begin{frame}
\frametitle{Architecture}
\textbf{Flynn's Taxonomy}

SISD -- Single Instruction, Single Data
(serial execution)

SIMD -- Single Instruction, Multiple Data
(MMX, SSE, AltiVec, 3dNow!, GPU)

MISD -- Multiple Instruction, Single Data 
(multiple filters)

MIMD -- Multiple Instruction, Multiple Data
(true multiprocessing)

\end{frame}

\subsection{Synchronization}
\begin{frame}
\frametitle{Synchronization}
%\begin{tabular}{cc}
%\textbf{Process 1} & \textbf{Process 2} \\
%$a = read(A)$    & $a = read(A)$ \\
%$a = a + 1$      & $b = read(B)$ \\
%$A = write(a)$   & $A = write(a +b)$
%\end{tabular}

\pause

Locking
\end{frame}


\begin{frame}[fragile]
\frametitle{Locks}
\begin{minted}[fontsize=\footnotesize]{csharp}
class Account {     // this is a monitor of an account
  long val = 0;
 
  public void Deposit(const long x) {
    lock (this) {   // only 1 thread at a time may execute this statement
      val += x;
    }
  }
 
  public void Withdraw(const long x) {
    lock (this) {
      val -= x;
    }
  }
}
\end{minted}
\end{frame}

\section{Libraries}
\subsection{multiprocessing}
\begin{frame}[fragile]
\frametitle{Python \texttt{multiprocessing}}
\begin{minted}[fontsize=\footnotesize]{python}
from multiprocessing import Pool
p = Pool()                  # initialize process pool
results = p.map(f, args)    # spawn processes
p.close()                   # close process pool 
\end{minted}
\end{frame}

\subsection{MapReduce}
\begin{frame}
\frametitle{MapReduce}
2004 Google framework\\
\texttt{\url{http://labs.google.com/papers/mapreduce.html}}\\
\textbf{APIs}: C++, C\#, Erlang, Java, OCaml, Perl, Python, PHP, Ruby, F\#, R

Six-stage pipeline:
\begin{itemize}
  \item input reader
  \item map function
  \item partition function
  \item comparison function
  \item reduce function
  \item output writer
\end{itemize}
\end{frame}

\begin{frame}[fragile]
\frametitle{MapReduce}
\begin{minted}[fontsize=\footnotesize]{java}
void map(String name, String document):
  // name: document name
  // document: document contents
  for word in document:
    EmitIntermediate(word, "1");
 
void reduce(String word, Iterator partialCounts):
  // word: a word
  // partialCounts: a list of aggregated partial counts
  int sum = 0;
  for pc in partialCounts:
    sum += ParseInt(pc);
  Emit(word, AsString(sum));
\end{minted}
\end{frame}

\begin{frame}
\frametitle{Resources}
\textbf{General}
\begin{itemize}
  \item \url{https://computing.llnl.gov/tutorials/parallel_comp/}
\end{itemize}

\textbf{MapReduce}
\begin{itemize}
  \item \url{http://labs.google.com/papers/mapreduce.html}
\end{itemize}
\end{frame}

\frame{\tableofcontents}

\frame{\titlepage}

\end{document}
